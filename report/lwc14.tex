\documentclass[a4paper]{article}
\usepackage[T1]{fontenc}
\usepackage{a4wide,times,hyperref,url,mathptm}
\usepackage[scaled=0.8]{beramono}
\begin{document}
\title{LWC'14: Rascal Submission}
\author{Tijs van der Storm \\ 
Centrum Wiskunde \&\ Informatica \\
\href{mailto:storm@cwi.nl}{\texttt{storm@cwi.nl}} / \href{http://twitter.com/tvdstorm}{\texttt{@tvdstorm}}
}
\maketitle

\section{The Rascal Language Workbench}


\url{http://www.rascal-mpl.org}

\cite{KlintSV09}

- Functional meta programming language featuring, extensible modules,
higher-order functions, powerful range of pattern matching operators,
comprehensions (map, set, list)

- Context-free grammars: modular, arbitrary, keyword reservation,
lookahead restrictions

- Rewrite rules and traversal primitives for tree transformation
(model-to-model).

- String templates for light-weight model-to-text transformation.

- Dynamic IDE support based on Eclipse through
IMP~\cite{CharlesFSDV09}: syntax highlighting, outliners,
hyperlinking, folding, error marking, hover documentation, building.


\section{QL and QLS base assignment}

The implementation of QL and QLS consists of syntax definition and
various analyses and transformations to implement semantic aspects
such as compilation, type checking and normalization.

The syntax of both languages is defined using Rascal's built-in
context-free grammars. Annotations in the grammar enable custom syntax
highlighting and code folding behavior in the QL/QLS editors.

Semantic analysis and compilation is implemented as transformations of
the syntax tree. Semantic analysis consists of name analysis (finding
a declaration for every used question name) and type checking
(ensuring that question are correctly used in expressions, and that
conditions are boolean expressions). In both cases, the tree is
decorated with errors and warnings (if any). These are shown at the
appropriate locations in the editor. When type checking QLS models,
the corresponding QL model is loaded first, to allow checking the
consistency of layout directives with respect to the questions in the
questionnaire.

Compilation (code generation) is realized using Rascal's string
templates. It is simply a traversal over the syntax tree (which is
assumed to have passed all semantic checks) which outputs JavaScript
code. 



\section{Scalability and teamwork}

\paragraph{Scalability}
Performance is an ongoing area of improvement in the Rascal
implementation. Currently, Rascal is fully interpreted which is
non-optimal for runtime performance. However, active development on
a compiler for Rascal will (hopefully) resolve many of the issues. 

The builtin parser of Rascal is based on GLL~\cite{GLLPTGEN}, an
algorithm with a worst-case bound of $O(n^3)$. However, on average
case grammars it is known to perform quite well (fast enough for
interactive use on Rascal source files itself). Ongoing research at
CWI aims to improve the performance of the parser as well.


\paragraph{Teamwork}
Since Rascal is a textual language workbench, any traditional version
control system can be used to facilate teamwork, on both language and
questionnaires. The user of QL/QLS can use the built-in support in
Eclipse for, e.g., Git, to show differences between files, apply
patches, or commit changes to version control. 

Nevertheless, because the QL/QLS DSL is relatively simple and
declarative, the Rascal implementation will contain semantic impact
analysis of changes to QL/QLS models. This will allow the modeler to
inspect the \textit{semantic} effects of a particular set of changes,
in order to validate that an applied patch does not introduce logic
that was unintended. An example is detection of question renames,
which does not affect the semantics of the questionnaire.

\section{Demonstration at LWC'14}



Scalability
- Performance trend of save (= parse/check/compile) over increasingly

parse 
parse + check
parse + check + compile

 
larger models. Using binary search questionnaire.


Team work

\bibliographystyle{abbrv}
\bibliography{lwc14}



\end{document}