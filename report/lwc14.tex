\documentclass[a4paper]{article}
\usepackage[T1]{fontenc}
\usepackage{a4wide,times,hyperref,url,mathptm}
\usepackage[scaled=0.8]{beramono}
\begin{document}
\title{LWC'14: Rascal Submission}
\author{Tijs van der Storm \\
Centrum Wiskunde \&\ Informatica \\
\href{mailto:storm@cwi.nl}{\texttt{storm@cwi.nl}} / \href{http://twitter.com/tvdstorm}{\texttt{@tvdstorm}}
\and Pablo Inostroza Valdera\\ 
Centrum Wiskunde \&\ Informatica \\
\href{mailto:pvaldera@cwi.nl}{\texttt{pvaldera@cwi.nl}}
}
\maketitle

\section{The Rascal Language Workbench}

Rascal is a meta programming language designed for transforming and
analyzing source code in the broad sense~\cite{KlintSV09}
(\url{http://www.rascal-mpl.org}). Embedded in Eclipse, Rascal
provides a versatile meta-environment for the design and
implementation of domain-specific language parsers, analyzers,
compilers and IDEs. Notable features include:
\begin{itemize}
\item Functional programming at the core, including support for
  higher-order functions, powerful pattern matching,
  comprehensions and extensible modules.
\item Built-in context-free grammars backed by general parsing. As a
  result, there is no restriction on grammar rules (e.g.,
  left-recursion is supported) and syntax definitions can be built in
  a modular fashion.
\item Rewrite rules and traversal primitives for tree transformation
  (model-to-model). String templates for lightweight model-to-text
  transformation.
\item Programmable IDE support based on Eclipse
  IMP~\cite{CharlesFSDV09}, which provides out-of-the-box support for
  syntax highlighting, code folding, identifier hyper linking, hover
  documentation, outlining, error marking and builders. 
\end{itemize}


\section{QL and QLS base assignment}

The implementation of QL and QLS consists of syntax definition and
various analyses and transformations to implement semantic aspects
such as type checking, normalization, and compilation. The
implementation will be similar to earlier implementations of QL and
QLS in Rascal~\cite{QLRKemi,Demoqles}, but for LWC'14 it will generate
code compatible with the reference solution~\cite{ReferenceImpl}.

The syntax of both languages is defined using Rascal's built-in
context-free grammars. Annotations in the grammar enable custom syntax
highlighting and code folding behavior in the QL/QLS editors. Semantic
analysis and compilation are transformations of the syntax tree.
Semantic analysis consists of name analysis (finding a declaration for
every used question name) and type checking (ensuring that question
are correctly used in expressions, and that conditions are boolean
expressions). In both cases, the tree is decorated with errors and
warnings (if any) which are shown at the appropriate locations in the
QL/QLS editors. When type checking QLS models, the corresponding QL
model is loaded first, to allow checking the consistency of question
placement and widget assignment. 

Compilation (code generation) consists of one or two phases, depending
on whether QLS is used. In any case, QL models are transformed to an
intermediate model. Then, this model is directly transformed to
JavaScript code using Rascal's string templates. Or, alternatively,
the intermediate model is first post-processed by the QLS compiler
which, based on a QLS model, reorders questions, changes widget types,
and adds styling information. After styling, the updated intermediate
model is rendered to JavaScript code just like before.



\section{Scalability and teamwork}

\paragraph{Scalability}
Performance is an ongoing area of improvement in the Rascal
implementation. Currently, Rascal is fully interpreted which could be
a problem for performance. However, active development of the compiler
for Rascal will significantly improve future performance. We hope to
show the new compiler at the LWC'14 workshop. 

The built-in parser of Rascal is based on GLL~\cite{GLLPTGEN}, an
algorithm with a worst-case complexity of $O(n^3)$. However, on
average case grammars it is known to often perform much better. For
instance, it is fast enough for interactive use in the editors for
Rascal source code itself. Ongoing research at CWI aims to improve the
performance of the parser as well.


\paragraph{Teamwork}
Since Rascal is a textual language workbench, any traditional version
control system can be used to facilitate teamwork, on both language and
questionnaires. The user of QL/QLS can use the built-in support in
Eclipse for, e.g., Git, to show differences between files, apply
patches, or commit changes to version control. 

While the traditional version control approach works well in practice,
sometimes one would like to have custom support, for instance, to
inspect differences between models. Rascal provides an interface for
hooking up custom ``diff'' support for your language. 


\section{Demonstration at LWC'14}

To demonstrate how well the Rascal implementation of QL scales, we
will use the Binary Search Questionnaire suggested in the reference
implementation~\cite{ReferenceImpl}. Running Rascal's benchmarking
library on parsing, parsing+checking, and parsing+checking+compilation
on increasing sizes of this questionnaire, allows us to plot out the
behavior of the runtime performance of the various components.

With respect to teamwork, we assume that the traditional
version-control work flow is well-known and does not need
demonstration. Instead we will show structural differencing support
for QL and QLS models.

\bibliographystyle{abbrv}
\bibliography{lwc14}



\end{document}